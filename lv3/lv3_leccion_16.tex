
% SPDX-License-Identifier: CC-BY-SA-4.0
% Copyright (c) 2025 Luis Alonso
%
% This file is part of luis_alonso/Notas-Japones.
%
% Licensed under Creative Commons Attribution-ShareAlike 4.0 International (CC BY-SA 4.0).
% You may share and adapt this material for any purpose,
% as long as you provide attribution and distribute adaptations under the same license.
%
% License: https://creativecommons.org/licenses/by-sa/4.0/
% Source:  https://git.lalonso.com/luis_alonso/Notas-Japones
%
% Note: Third-party content (if any) is excluded from this license and is marked where used.

\documentclass[a4paper,lualatex,ja=standard,base=10pt]{bxjsarticle}

\setmainjfont{Noto Serif CJK JP}

\usepackage{luatexja-fontspec}
\usepackage{graphicx}
\usepackage{fancyhdr}
\usepackage{luatexja-ruby}
\usepackage{tikz}

\title{\ruby{第|1|6|課}{だい|||か} LV3}
\author{アロンゾ ルイス}

% --- Page style ---
\pagestyle{fancy}
\fancyhf{} % clear header/footer

% Footer: name + copyright on left, page number on right
\fancyfoot[L]{Luis Alonso\ \copyright\ \the\year}
\fancyfoot[R]{\thepage}

% Optional: remove header rule and keep a footer rule
\renewcommand{\headrulewidth}{0pt}
\renewcommand{\footrulewidth}{0.4pt}

% Increase space reserved for header area
\setlength{\headheight}{0pt}
\setlength{\headsep}{6pt} % distance from header to text

\fancypagestyle{plain}{%
  \fancyhf{}
  \fancyfoot[L]{Luis Alonso\ \copyright\ \the\year}
  \fancyfoot[R]{\thepage}
  \renewcommand{\headrulewidth}{0pt}
  \renewcommand{\footrulewidth}{0.4pt}
}

\begin{document}
\setcounter{page}{1}

\raggedbottom
\maketitle{}
\thispagestyle{fancy}

% Content
\section{Usar ropa y accesorios}

\begin{figure}[h]
\centering
\begin{tikzpicture}
  % Put your image file name here:
  \node[anchor=south west, inner sep=0] (img) at (0,0)
    {\includegraphics[scale=0.25]{assets/smartphone_schoolgirl_walk.png}};

  % Normalized coordinate system: (0,0)=bottom-left, (1,1)=top-right of the image
  \begin{scope}[x={(img.south east)}, y={(img.north west)}]

    % --- target points (adjust these if you want) ---
    \coordinate (upper) at (0.62,0.55); % top/upper body
    \coordinate (lower) at (0.25,0.24); % lower body
    \coordinate (bag)   at (0.95,0.25); % bag

    % --- label positions (outside the image) ---
    \node[anchor=east] (Lupper) at ( 1.06,0.72) {きます};
    \node[anchor=east] (Llower) at (-0.06,0.34) {はきます};
    \node[anchor=west] (Lbag)   at ( 1.06,0.25) {もちます};

    % --- arrows ---
    \draw[->, thick] (Lupper.west) -- (upper);
    \draw[->, thick] (Llower.east) -- (lower);
    \draw[->, thick] (Lbag.west)   -- (bag);

  \end{scope}
\end{tikzpicture}
\end{figure}

\textbf{Verbos para indicar vestir ropa:}

\begin{tabular}{@{} l @{\ :\ } p{0.75\linewidth} @{}}
きます (\ruby{着|ます}{き|})     & Para ropa que se usa por arriba de la cintura. \\
はきます (\ruby{履|きます}{は|}) & Para ropa que se usa por debajo de la cintura. \\
もちます (\ruby{持|ちます}{も|}) & Para cosas que se llevan en las manos. \\
\end{tabular}

\end{document}

