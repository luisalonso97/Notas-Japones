% SPDX-License-Identifier: CC-BY-SA-4.0
% Copyright (c) 2026 Luis Alonso
%
% This file is part of luis_alonso/Notas-Japones.
%
% Licensed under Creative Commons Attribution-ShareAlike 4.0 International (CC BY-SA 4.0).
% You may share and adapt this material for any purpose,
% as long as you provide attribution and distribute adaptations under the same license.
%
% License: https://creativecommons.org/licenses/by-sa/4.0/
% Source:  https://git.lalonso.com/luis_alonso/Notas-Japones
%
% Note: Third-party content (if any) is excluded from this license and is marked where used.

\documentclass[a4paper,lualatex,ja=standard,base=10pt]{bxjsarticle}
\usepackage{setup}

\title{\ruby{第|16|課}{だい||か} LV3}
\author{アロンゾ ルイス}

\begin{document}
\setcounter{page}{1}

\maketitle{}
\thispagestyle{fancy}

\section{Usar ropa y accesorios}

\begin{figure}[h]
\centering
\begin{tikzpicture}
  % Image
  \node[anchor=south west, inner sep=0] (img) at (0,0)
    {\includegraphics[scale=0.25]{assets/smartphone_schoolgirl_walk.png}};

  % Normalized coordinate system: (0,0)=bottom-left, (1,1)=top-right of the image
  \begin{scope}[x={(img.south east)}, y={(img.north west)}]

    % --- target points (adjust these if you want) ---
    \coordinate (upper) at (0.62,0.55); % top/upper body
    \coordinate (lower) at (0.25,0.24); % lower body
    \coordinate (bag)   at (0.95,0.25); % bag

    % --- label positions (outside the image) ---
    \node[anchor=east] (Lupper) at ( 1.06,0.72) {きます};
    \node[anchor=east] (Llower) at (-0.06,0.34) {はきます};
    \node[anchor=west] (Lbag)   at ( 1.06,0.25) {もちます};

    % --- arrows ---
    \draw[->, thick] (Lupper.west) -- (upper);
    \draw[->, thick] (Llower.east) -- (lower);
    \draw[->, thick] (Lbag.west)   -- (bag);

  \end{scope}
\end{tikzpicture}
\caption{Uso de los verbos para usar ropa y accesorios.}
\end{figure}

\begin{PlusHeaderTable}{Verbos para usar ropa}
  Pieza de ropa & を & \ruby{着|ます}{き|} \\
  Pieza de ropa & を & \ruby{履|きます}{は|} \\
  Pieza de ropa & を & \ruby{持|ちます}{も|} \\
\end{PlusHeaderTable}

\begin{VocabTable}[0.75\linewidth][:]
  きます   (\ruby{着|ます}{き|})   & Para ropa que se usa por arriba de la cintura. \\
  はきます (\ruby{履|きます}{は|}) & Para ropa que se usa por debajo de la cintura. \\
  もちます (\ruby{持|ちます}{も|}) & Para cosas que se llevan en las manos.         \\
\end{VocabTable}
\medskip

\textbf{Tallas de ropa:}

\begin{VocabTable}[0.45\linewidth][-][l,l,P]
  S (エズ) & サイズ & Talla chica   \\
  M (エム) & サイズ & Talla mediana \\
  L (エル) & サイズ & Talla grande  \\
\end{VocabTable}
\medskip

\Needspace{2\baselineskip}
\textbf{Ejemplos de uso:}
\begin{itemize}
  \item コートを\ruby{着|ます}{き|} - (Usar un abrigo)
  \item ジーンズを\ruby{履|きます}{は|} - (Usar jeans, pantalones de mezclilla)
  \item くつを\ruby{履|きます}{は|} - (Usar zapatos)
  \item くつしたを\ruby{履|きます}{は|} - (Usar calcetines)
\end{itemize}

\section{Colores \ruby{色}{いろ}}

\begin{paracol}{2}
  \begin{itemize}
    \item \ruby{黄|色}{き|いろ}: Amarillo
    \item \ruby{白}{しろ}: Blanco
    \item \ruby{茶|色}{ちゃ|いろ}: Café
    \item \ruby{赤}{あか}: Rojo
    \item \ruby{紫}{むらさき}: Púrpura
  \end{itemize}
  \switchcolumn
  \begin{itemize}
    \item \ruby{黒}{くろ}: Negro
    \item オレンジ: Naranja
    \item ピンク: Rosa
    \item \ruby{緑}{みどり}: Verde
    \item グレー: Gris
  \end{itemize}
\end{paracol}

\section{こそあど}

El sistema こそあど es un conjunto de demostrativos del japonés que se usan para
señalar personas, cosas, lugares o direcciones, según la distancia psicológica
o física entre el hablante, el oyente y el objeto.

\begin{table}[h!]
  \centering
  \renewcommand{\arraystretch}{1.4}
  \begin{tabular}{|p{2.5cm}|p{2.5cm}|p{2.5cm}|p{2.5cm}|p{2.5cm}|}
    \hline
    \textbf{Categoría} & \textbf{Cerca del hablante (こ)} & \textbf{Cerca del oyente (そ)} & \textbf{Lejos de ambos (あ)} & \textbf{Interrogativo (ど)} \\
    \hline
    Cosa & これ & それ & あれ & どれ \\
    \hline
    Determinante + sus. & この & その & あの &  \\
    \hline
    Lugar & ここ & そこ & あそこ & どこ \\
    \hline
    Dirección & こちら & そちら & あちら & どちら \\
    \hline
    Persona & この人 & その人 & あの人 & どの人 \\
    \hline
    Tipo / manera & こんな & そんな & あんな & どんな \\
    \hline
  \end{tabular}
  \caption{Sistema de demostrativos japoneses こそあど}
\end{table}

Los demostrativos これ・それ・あれ se usan solos, mientras que この・その・あの deben ir siempre
seguidos de un sustantivo.

\end{document}

