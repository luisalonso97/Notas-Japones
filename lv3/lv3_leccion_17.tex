% SPDX-License-Identifier: CC-BY-SA-4.0
% Copyright (c) 2026 Luis Alonso
%
% This file is part of luis_alonso/Notas-Japones.
%
% Licensed under Creative Commons Attribution-ShareAlike 4.0 International (CC BY-SA 4.0).
% You may share and adapt this material for any purpose,
% as long as you provide attribution and distribute adaptations under the same license.
%
% License: https://creativecommons.org/licenses/by-sa/4.0/
% Source:  https://git.lalonso.com/luis_alonso/Notas-Japones
%
% Note: Third-party content (if any) is excluded from this license and is marked where used.

\documentclass[a4paper,lualatex,ja=standard,base=10pt]{bxjsarticle}
\usepackage{setup}

\title{\ruby{第|17|課}{だい||か} LV3}
\author{アロンゾ ルイス}

\begin{document}
\setcounter{page}{1}

\maketitle{}
\thispagestyle{fancy}

\section{Negar adjetivos al pasado} \medskip

Los adjetivos な e い se conjugan al no pasado de dos maneras distintas.
\medskip

\textbf{Adjetivos い en pasado}

\begin{VocabTable}[0.35\linewidth][→][P,P]
Pos. Adj.いです。     & Pos. Adj.かったです。\\
Neg. Adj.くないです。 & Neg. Adj.くなかったです。\\
\end{VocabTable}
\medskip

\textbf{Adjetivos な al pasado}

\begin{VocabTable}[0.35\linewidth][→][P,P]
Pos. Adj.なです。       & Pos. Adj.なでした。\\
Neg. Adj.じゃないです。 & Neg. Adj.じゃなかったです。\\
\end{VocabTable}
\medskip
\medskip

\textbf{Ejemplos:} \medskip

\begin{VocabTable}[0.35\linewidth][→][P,P]
  \ruby{嬉|しいかったです。}{うれ|} & \ruby{嬉|しいくなかったです。}{うれ|}\\
\end{VocabTable}

\section{どにも y \ruby{何|も}{なに|}} \medskip

Son palabras usadas para indicar que no se fue a ningún lugar o no se hizo nada
en particular.

\end{document}

