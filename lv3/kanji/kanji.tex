% SPDX-License-Identifier: CC-BY-SA-4.0
% Copyright (c) 2025 Luis Alonso
%
% This file is part of luis_alonso/Notas-Japones.
%
% Licensed under Creative Commons Attribution-ShareAlike 4.0 International (CC BY-SA 4.0).
% You may share and adapt this material for any purpose,
% as long as you provide attribution and distribute adaptations under the same license.
%
% License: https://creativecommons.org/licenses/by-sa/4.0/
% Source:  https://git.lalonso.com/luis_alonso/Notas-Japones
%
% Note: Third-party content (if any) is excluded from this license and is marked where used.

\documentclass[a4paper,lualatex,ja=standard,base=10pt]{bxjsarticle}

\title{\ruby{漢|字}{かん|じ} LV3}
\author{アロンゾ ルイス}

\setmainjfont{Noto Serif CJK JP}
\usepackage{luatexja-fontspec}
\usepackage{kanjicard} % Custom Kanji card
\usepackage{fancyhdr}

% --- Page style ---
\pagestyle{fancy}
\fancyhf{} % clear header/footer

% Footer: name + copyright on left, page number on right
\fancyfoot[L]{Luis Alonso\ \copyright\ \the\year}
\fancyfoot[R]{\thepage}

% Optional: remove header rule and keep a footer rule
\renewcommand{\headrulewidth}{0pt}
\renewcommand{\footrulewidth}{0.4pt}

% Increase space reserved for header area
\setlength{\headheight}{0pt}
\setlength{\headsep}{6pt} % distance from header to text

\fancypagestyle{plain}{%
  \fancyhf{}
  \fancyfoot[L]{Luis Alonso\ \copyright\ \the\year}
  \fancyfoot[R]{\thepage}
  \renewcommand{\headrulewidth}{0pt}
  \renewcommand{\footrulewidth}{0.4pt}
}

\begin{document}
\setcounter{page}{1} % start page number here

\raggedbottom
\maketitle{}
\thispagestyle{fancy}

\begin{kanjicard}{口}{くち}{コウ/ク}
  \KExample{口}{ぐち}{Boca}
  \KExample{東口}{ひがし・ぐち}{Salida este}
  \KExample{入口}{いり・ぐち}{Entrada}
  \KExample{出口}{で・ぐち}{Salida}
\end{kanjicard}

\begin{kanjicard}{東}{ひがし}{トウ}
  \KExample{東口}{ひがし・ぐち}{Puerta este}
  \KExample{東京}{とう・きょう}{Tokio}
  \KExample{関東}{かん・とう}{Kanto}
\end{kanjicard}

\begin{kanjicard}{西}{にし}{セイ/サイ}
  \KExample{西口}{にし・ぐち}{Salida oeste}
  \KExample{西洋}{せい・よう}{Países Occidentales}
  \KExample{関西}{かん・さい}{Región de Kansai}
\end{kanjicard}

\begin{kanjicard}{南}{みなみ}{メン}
  \KExample{南口}{みなみ・ぐち}{Salida sur}
  \KExample{南米}{なん・べい}{América del sur}
  \KExample{南北}{なん・ぼく}{Sur y norte}
\end{kanjicard}

\begin{kanjicard}{北}{きた}{ホク/ホッ}
  \KExample{北口}{きた・ぐち}{Salida norte}
  \KExample{北米}{ほく・べい}{América del norte}
  \KExample{北海道}{ほっ・かい・どう}{Hokkaido}
\end{kanjicard}

\begin{kanjicard}{町}{まち}{チョウ}
  \KExample{町}{まち}{Pueblo, ciudad}
  \KExample{町内}{ちょう・ない}{Barrio, vecindad}
\end{kanjicard}

\begin{kanjicard}{寺}{てら}{ジ}
  \KExample{お寺}{お・てら}{Templo budista}
  \KExample{寺社}{じ・しゃ}{Templos y santuarios}
\end{kanjicard}

\begin{kanjicard}{店}{みせ}{テン}
  \KExample{店}{みせ}{Tienda}
  \KExample{書店}{しょ・てん}{Libreria}
\end{kanjicard}

\begin{kanjicard}{停}{と・まります}{テイ}
  \KExample{停まります}{と・まります}{Parada (de transporte)}
  \KExample{バス停}{バズ・てい}{Parada de autobus}
\end{kanjicard}

\begin{kanjicard}{枚}{}{マイ/バイ}
  \KExample{四枚}{よん・まい}{Cuatro objetos delgados planos}
  \KExample{数枚}{すう・まい}{Varios objetos delgados planos}
\end{kanjicard}

\begin{kanjicard}{個}{}{コ/カ}
  \KExample{二個}{に・こ}{Dos objetos pequeños}
  \KExample{個人}{こ・じん}{Individuo, privado}
\end{kanjicard}

\begin{kanjicard}{冊}{}{サツ/サク}
  \KExample{三冊}{さん・さつ}{Tres libros, tres tomos}
  \KExample{小冊子}{しょう・さつ・し}{Folleto}
\end{kanjicard}

\begin{kanjicard}{買}{か・います}{}
  \KExample{買います}{か・います}{Comprar}
  \KExample{買い物}{か・いもの}{Compras}
\end{kanjicard}

\begin{kanjicard}{金}{かね}{キン}
  \KExample{お金}{お・かね}{Dinero}
  \KExample{金}{きん}{Oro}
  \KExample{金曜日}{きん・よう・び}{Viernes}
  \KExample{金田}{かね・だ}{Kaneda (apellido)}
\end{kanjicard}

\begin{kanjicard}{円}{}{エン}
  \KExample{円}{えん}{Yen}
\end{kanjicard}

\begin{kanjicard}{百}{}{ヒアク/ビアク/ピアク}
  \KExample{二百}{に・ひゃく}{200}
  \KExample{三百}{さん・びゃく}{300}
  \KExample{六百}{ろ・ぴゃく}{600}
  \KExample{八百}{はっ・ぴゃく}{800}
\end{kanjicard}

\begin{kanjicard}{千}{ち}{セン/ゼン}
  \KExample{千円}{せん・えん}{1000 yenes}
  \KExample{三千}{さん・ぜん}{3000}
\end{kanjicard}

\begin{kanjicard}{万}{}{マン/バン}
  \KExample{一万円}{いち・まん・えん}{1000 yenes}
  \KExample{万国}{ばん・こく}{Todos los países}
  \KExample{万歳}{ばん・ざい}{Banzai}
\end{kanjicard}

\begin{kanjicard}{楽}{楽・しい}{ガク/ラク/ゴウ}
  \KExample{楽しい}{たの・しい}{Divertido}
  \KExample{楽}{らく}{Comodo, sencillo}
  \KExample{音楽}{おん・がく}{Música}
\end{kanjicard}

\begin{kanjicard}{嬉}{うれ・しい}{}
  \KExample{嬉しい}{うれ・しい}{Féliz, alegre}
\end{kanjicard}

\begin{kanjicard}{良}{い・い/よ・い}{リョウ}
  \KExample{良い}{い・い}{Está bien}
  \KExample{良心}{りょう・しん}{Conciencia}
\end{kanjicard}

\begin{kanjicard}{来}{き・ます}{ライ}
  \KExample{来ます}{き・ます}{Venir}
  \KExample{来週}{らい・しゅう}{Próxima semana}
  \KExample{未来}{み・らい}{Futuro}
\end{kanjicard}

\begin{kanjicard}{会}{あ・います}{カイ}
  \KExample{会います}{あ・います}{Encontrarse}
  \KExample{会議}{かい・ぎ}{Junta, reunión}
  \KExample{大会}{たい・かい}{Competencia}
\end{kanjicard}

\begin{kanjicard}{休}{やす・みます}{キュウ}
  \KExample{休みます}{やす・みます}{Descansar}
  \KExample{休憩}{きゅう・けい}{Descanso, pausa}
  \KExample{休日}{きゅう・じつ}{Día libre}
\end{kanjicard}

\begin{kanjicard}{本}{もと}{ホン}
  \KExample{本}{ほん}{Libro}
  \KExample{日本}{に・ほん}{Japón}
\end{kanjicard}

\begin{kanjicard}{京}{みやこ}{キョウ/ケイ/キン}
  \KExample{東京}{とう・きょう}{Tokio}
  \KExample{京都}{きょう・と}{Kioto}
\end{kanjicard}

\end{document}

